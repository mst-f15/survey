\subsubsection{Triangulation}\label{triangulation}
Triangulation is the technique of using distance measurements to a number of reference points to calculate the location of an object.
Typically the distance is measured using the signal strength between object and the reference points or the time of arrival of the signal.
A model is then used to map the distance measurements to a 2D or 3D position of the object.\cite[Section 4.1]{rfidreview}

\paragraph{SpotON}
An implementation of triangulation is the SpotON system\cite{spoton}.
SpotON uses several base stations that are all connected to a central server.
The server aggregates the received positions, triangulates the data and sends the resulting position to clients. \cite{spoton}

The algorithm used for triangulating the positions is a hillclimbing algorithm as presented in \citet[Section 3.3.1]{spoton}.
The algorithm starts from a arbitrary position and searches for the real position of the object by calculating the theoretical signal strengths for adjacent positions in a coordinate system.
The theoretical signal strengths are then compared to the observed signal strengths, and the point with the smallest error is used as the next point of origin.
This process is repeated until the point does not move \cite{spoton}.

\paragraph{Criticism}
This method of triangulation is dependent on the calibration of the model of each base station.
It is therefore prone to errors when the environment changes.
Moving the base stations or even the surrounding objects will influence the precision of the calculations.

Also the hillclimbing method can be a problem in big buildings or big rooms, if there is no way to provide a good guess for the initial position.

