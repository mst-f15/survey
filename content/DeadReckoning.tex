Dead reckoning (DR) \cite{DR} is a navigation technique to estimate the positioning of a moving object.
DR make use of previous know positions to predict an estimated positioning of the object after a given time.
The known positions can be used to estimate the direction and velocity of the object over elapsed time if they are not already known.
Using the data it is possible to calculate a displacement from the last known position after the time has passed.
The estimated displacement of the object is easily influences by sources of errors.
The object might have changed direction or stopped moving.
This may result in a wrong estimate which may drift over time.

Data used for DR can be obtained from internal sensors that the object is carrying or external sensors which is observing the object.
An example which include both internal and external sensors could be smartphones.
Smartphones can be self-contained using built-in components such as gyroscope, accelerometer and compass.
They might also use the built-in GPS to communicate with satellites which provides the smartphones with external data.