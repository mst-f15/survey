\subsubsection{RFID}
RFID (Radio Frequency Identification) provides a way to tag items by means of a cheap tag that can then later be read with a RFID reader.

A RFID tag is a small object often in the form of a sticker, that can hold data and send it to a reader upon request.
A tag contains a microchip containing the data and an antenna for communication with a reader.
Tags come in different variants regarding power source.
There exists both passive and active RFID tags.\cite{rfidreview}
A RFID reader is used to read and write data to RFID tags.

\paragraph{Limitations}
The passive tags have no power source, and therefore relies on electromagnetic energy from the reader.
This limits the abilities of the tag to only containing a small amount of information, but they are very cheap.
Also the range of these tags are limited. \cite{rfidreview}

Active tags have an internal battery.
This enables longer read ranges, onboard memory and even processing capacities.
The battery shortens the lifespan of the tag, which is typically between 5 and 10 years, and they are more expensive than the passive counterpart. \cite{rfidreview}