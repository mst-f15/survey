Each building will have a unique indoor magnetic field, made unique by the construction (concrete, steel beams etc.) of each building.
Each individual location within a building will be affected by the aforementioned parameters and will therefore have a unique signature, depending on how the magnetic field is at that exact location.

There have been very few attempts at utilizing these properties of the unique magnetic fields for a building, most noteworthy is \cite{geomagnetism}.
In \citet{geomagnetism} experiments are carried out in two buildings with both a robot and a human-wearable compass-construction.
Simple electric compasses are used for collecting data in a full 360 degree radius, while the least root mean square is calculated and nearest neighbour search is performed to match obtained location data with obtained map data.
Initial results with the robot were $err_{mean} = 3.05 m$, $err_{sd} = 4.09 m$ and $err_{max} = 15 m$, where 70 \% of the predictions had errors less than 2 m.
The second experiment with a human-worn device (with 4 compasses) yielded an overall result of $<1m$ error for 75.7 \% of the results.
Two overall methods of improving the results are suggested, where one is adding an additional hardware component for assisting localization and the other being an added algorithmic.
By adding this filtering (by limiting search space based on previous known location) a $<1 m$ accuracy for 86.6 \% of predictions is obtained.
Finally experiments are carried out to show that the magnetic field changes little over time (very little noticeable difference with 6 months between) and that furniture and electrical equipment have little impact on measurements.


