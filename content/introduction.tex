% Spending more time indoors
% GPS insufficient
% Current map making methods
% Current, closed, localization/navigation solutions

In this section, a brief overview of the currently available commercial solutions in regards to indoor mapping and localization will be presented, without going into the inner workings.

Google Maps Indoor already provides a publicly available method of making indoor maps (adding them to Google Maps), by uploading and fitting a schema on top of Google Maps \cite{google_maps_indoor}.
Additionally, Cartographer was recently introduced; a backpack with sensor equipment using SLAM (Simultaneous Localization and Mapping) to continuously map while moving through a venue \cite{cartographer}.
The method for localization is a fused one, using sensors, GPS, wifi, and selected level \cite[Slide 45]{indoor_maps_google_slides}.

Micello, similar to Google, accepts floor plans to generate venue maps \cite{micello}.
Micello provides navigation within a venue, however, it does not provide any method of localization, while it is compatible with several localization providers.

Connexient provides professional indoor mapping and navigation solutions.
While Connexient does all the initial work with the creation of the map, the plan is to provide tools to keep them updated in-house \cite{connexient_indoor_map}.
Localization is handled by a fusion of BLE (Bluetooth Low Energy) beacons, RSS (Received Signal Strength), compass, and accelerometer \cite{connexient_indoor_pos}.
