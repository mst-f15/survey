% Spending more time indoors
% GPS insufficient
% Current map making methods
% Current, closed, localization/navigation solutions

People spend in average 87 \% of their time indoors\cite{time_spend_indoor}, yet the most well-versed mapping is only usable outside.
This includes both creation of maps and localization/navigation.

Google Project Ground Truth\cite{googleio_ground_truth} already cover most of the outdoors, and are now focusing on moving indoors\cite{googleio_indoor_maps}\cite{indoor_maps_google_slides}.
Methods of creating the maps include uploading and fitting of custom floor plans on top of Google Maps.
Another method, recently announced by Google, is the usage of a sensor-backpack using SLAM, called Cartographer\cite{cartographer}.
Other mention-worthy public solutions that are also available for creating indoor maps are Bing Maps\cite{bingmaps} and Micello\cite{micello}.

What is missing now, is proper indoor localization, to use for tracking and navigation.
GPS\cite{gps} is both effective and efficient outdoors, however, it is near unusable in the indoors, as both availability and precision are not suited for indoor usage.

There are several public and commercial solutions with regards to indoor localization, however, the inner workings of these are not available.
These solutions include Google Indoor Maps, which uses a fusion of WiFi RSS (Received Signal Strength) and mobile sensors, obtaining a 7 meter accuracy\cite{googleio_indoor_maps}.
Connexient is a professional solution, which uses BLE (Bluetooth Low Energy) beacons in combination with WiFi RSS, mobile compass and accelerometer readings, to obtain a claimed accuracy of 1-2 meters\cite{connexient_indoor_pos}.
Nextome is another professional solution, claiming to obtain a 0.5 meter accuracy by using iBeacons\cite{ibeacon} without the use of any net-based services, but only \textit{''[...]The innovative algorithm of NEXTOME uses advanced techniques of artificial intelligence to receive, categorize and analyze signals and identify the user’s position in the room.[...]''}\cite{nextome_indoor_pos}.
Another solution for indoor  mapping and localization is IndoorAtlas, which uses magnetic positioning alone (with the option of supplementing with WiFi/Bluetooth), while claiming \textit{''industry-leading accuracy for free''}\cite{indooratlas_features}.

As mentioned, all these solutions are very secretive, and the actual inner workings and algorithms are not public available.
Therefore the current general solutions will be looked into.