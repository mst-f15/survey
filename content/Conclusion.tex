\section{Conclusion}
We have in this survey focused on indoor mapping, where we have studied different solutions.
There are multiple solutions using different technologies: WLAN, Bluetooth and RFID and methods: Triangulation, Proximity analysis, Fingerprinting and Hybrid.

The different radio-based technologies and localization methods are prone to errors due to changing environment and obstacles.
Technologies such as WLAN are widely common thus easily available to use for indoor mapping while Bluetooth and RFID are on the other hand not commonly available and are thus costly to implement.
The triangulation, proximity analysis and fingerprinting are different methods that uses mapping and reference points to estimate the position of the object.

The use of magnetic fields is, in theory, a good alternative to WiFi-based localization, as it is completely independent of fixed-location hardware, such as WiFi hotspots.
However, it does share the same limitation as the other fingerprinting methods, in the requirements of a pre-made offline map before online localization can be performed.
As can be seen, it is still relatively new and unexplored, and thus it requires further research before it can be completely deemed useful.
