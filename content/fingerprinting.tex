\subsubsection{Fingerprinting}\label{fingerprinting}
This section is based on \citet{fingerprinting_slides} and \citet{fingerprinting}.
Location fingerprinting is a positioning technique that measures the strength of different signals at a given position.
This measurement is then the fingerprint for that location.

This section will explain the two phases of fingerprinting and how they are stored.
\paragraph{Offline phase}
In the offline phase a site-survey is performed, where a radio map is being build.
This means measuring the strength of the different signals at all points of interest and storing them in a radio map.
A radio map is a database that contains received signal strength indicator(RSSI) values for different positions throughout a building.
A position is represented by a triplet $(x,y,z)$ where $x$ and $y$ represent the 2-D coordinates and $y$ the height of the sensor that measures the signal.
The signals are then stored for instance as a vector for each position.

For instance if one is using WiFi for localization, when measuring a specific location the RSSI from each access point is saved and that is the fingerprint for that location.

\paragraph{Online phase}
The online phase consists of collecting the RSSI values from the client and look them up in the radio map and return the position.
But most of the time the values collected from the client do not match any values on the radio map.
For that one needs to use a position estimation method to get the best match.
The simplest is to use the Nearest Neighbour algorithm, that will return the position with the shortest vector distance.
Other algorithms useful are Weighed K Nearest Neighbour and K Nearest Neighbour.

