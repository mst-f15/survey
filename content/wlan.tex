\subsection{What is WLAN}
WLAN stands for Wireless Local Area Network.
WLAN provides wireless network access by using high frequency radio waves.
The network often provides access to the internet.
The most used WLAN standard is the IEEE 802.11 standard\cite{ieee_wifi_standard}, also known as, WiFi.
WiFi mostly uses 2.4 GHz frequencies for radio waves\cite{ieee_wifi_standard}.

\subsection{WiFi for indoor positioning}
WiFi is used for indoor positioning because of its availability
\cite{indoor_maps_google_slides}\cite{improving_wifi_using_bluetooth}.
This is the only reason for using WiFi, because the other methods available, bluetooth and RFID, are more expensive to set up \cite{improving_wifi_using_bluetooth}.
The next section will describe the challenges one encounters when using WiFi for indoor positioning.

\subsection{Challenges}
This section is based on section 2 from \cite{computationally_efficient_wlan}.
The standard location determination is done using AOA(Angle of Arrival) and the time difference(Time  of Arrival). But when measuring these things indoor there is the following noise:
\begin{itemize}
	\item Interference by other devices using the 2,4 GHz frequency - cordless phones, bluetooth devices etc.
	\item The 2,4 GHz frequency is a resonant frequency of water - this means as humans mostly consist of water - so that they interfere with the signal as well.
\end{itemize}
An RF signal travels 3 meters in 10 nanoseconds - but because of the different noise it is difficult to determine the TOA anyways

These noise factors mean that empirically manual collected data is used. The drawback that it is takes a long time and that the data changes because of the amount of people in the venue.

Another challenge to take into account is the power usage.
A mobile phones WiFi interface consumes 5 times the power of what the bluetooth uses\cite{bluetooth_basics}.

Or if one of the access points is taken down or its position is changed the whole radiomap needs to be updated\cite{improving_wifi_using_bluetooth}.