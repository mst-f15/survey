With WiFi being so inaccurate and being so sensitive to noise, see section \ref{wifi_challenges}.
Some studies have suggested doing indoor mapping by merging technologies such as WiFi and bluetooth \cite{hybrid_wifi_bluetooth} \cite{fusion_wifi_bluetooth}.

\subsection{Fusion}
The fusion model suggested by \citet{fusion_wifi_bluetooth} uses WiFi and bluetooth.


\subsection{Hybrid}
The hybrid model suggested by \citet{hybrid_wifi_bluetooth} uses WiFi and bluetooth.
It argues that because the availability of WiFi and the cost of deploying bluetooth devices a hybrid model is the best solution in most cases for improving the accuracy.
The model uses the WiFi already on hand and adds bluetooth devices in strategic positions.
The bluetooth devices are used for partitioning the indoor space.
So for instance when used in a mall, one could add a bluetooth device to each entrance of every shop.
This way one would be sure in which shop one is in.
\textbf{Måske lidt skørt at komme med et eksempel..........}
Furthermore it solves the problem of adjacent rooms/shops/partitions - with bluetooth devices it can be seen where the subject has entered.