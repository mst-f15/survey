\documentclass[12pt,a4paper]{article}
\usepackage[top=2.5cm,bottom=2.5cm,left=2.5cm,right=2.5cm]{geometry}
\usepackage{setspace}
\onehalfspacing
\usepackage[utf8]{inputenc}
\usepackage{hyperref}
\usepackage[numbers]{natbib}
\usepackage{tikz}
\usepackage{todonotes}
\usepackage{cleveref}


\begin{document}
\begin{center}
\bigskip\bigskip
{\Large\bf Google Indoor Maps}
\bigskip\bigskip

% Make sure all student names, email addresses and study names are correct!

Bruno Thalmann, bthalm11@student.aau.dk, SW8\\
Henrik Haxholm, hhaxho11@student.aau.dk, SP8\\
Mikael Elkiær Christensen, michri11@student.aau.dk, SW8\\
Mikkel Sandø Larsen, milars11@student.aau.dk, SW8\\
Stefan Marstrand Getreuer Micheelsen, smiche11@student.aau.dk, SW8

\bigskip
\begin{abstract}
\textbf{This survey explores the current technologies and methods currently available for indoor localization.
Since Google Indoor Maps do not provide a sole answer to this problem, other sources are explored.
The main technologies currently used are radio-based, such as WiFi, Bluetooth and RFID, where related methods are explored.
Finally and briefly, dead reckoning, along with the relatively unexplored geo-magnetism-based localization, is also explored.
Finally, as a conclusion, the presented technologies and methods are compared, and it is suggested what might be interesting to focus on in the future.}
\end{abstract}

\thispagestyle{empty}
\end{center}


\titlepage



\thispagestyle{plain}
\pagenumbering{arabic}

% NOTE: From here on, you have SIX pages EXCLUDING the references.

\section{Introduction}
% Spending more time indoors
% GPS insufficient
% Current map making methods
% Current, closed, localization/navigation solutions

People spend most of the time indoors, yet the most well-versed mapping is only usable outside.
This includes both creation of maps and localization/navigation.

Google Project Ground Truth\cite{googleio_ground_truth} already cover most of the outdoors, and are now focusing on moving indoors \cite{googleio_indoor_maps}\cite{indoor_maps_google_slides}.
Methods of creating the maps include uploading and fitting of custom floor plans on top of Google Maps.
Another method, recently announced by Google, is the usage of a sensor-backpack using SLAM, called Cartographer \cite{cartographer}.
Other mention-worthy public solutions that are also available for creating indoor maps are Bing Maps\cite{bingmaps} and Micello\cite{micello}.

What is missing now, is proper indoor localization, to use for tracking and navigation.
GPS is both effective and efficient outdoors, however, it is near unusable in the indoors, as both availability and precision are not suited for indoor usage.

There are several public and commercial solutions with regards to indoor localization, however, the inner workings of these are not available.
These solutions include Google Indoor Maps, which uses a fusion of WiFi RSS and mobile sensors, obtaining 7 meter accuracy \cite{googleio_indoor_maps}.
Connexient is a professional solution, which uses a fusion of BLE (Bluetooth Low Energy) beacons in combination with WiFi RSS, mobile compass and accelerometer readings, claiming accuracy of 1-2 meters \cite{connexient_indoor_pos}.
Nextome is another professional solution, which uses iBeacons\cite{ibeacon}, claiming a 0.5 meter accuracy without the use of any net-based services, but only \textit{''[...]The innovative algorithm of NEXTOME uses advanced techniques of artificial intelligence to receive, categorize and analyze signals and identify the user’s position in the room.[...]''} \cite{nextome_indoor_pos}.

As mentioned, all these solutions are very secretive, and the actual inner workings and algorithms are not known.
Therefore the current general solutions will be looked into.


\section{Radio-based}
This section contains the technologies that typically are used for indoor localization and the methods used for doing so.
\subsection{Technologies}
This section describes three radio-based technologies, \textit{WLAN}, \textit{Bluetooth} and \textit{RFID}, that are used for indoor localization.
\subsubsection{WLAN}\label{wifi}
WLAN stands for Wireless Local Area Network.
WLAN provides wireless network access by using high frequency radio waves.
The network often provides access to the internet.
The most used WLAN standard is the IEEE 802.11 standard\cite{ieee_wifi_standard}, also known as WiFi.
WiFi mostly uses 2.4 GHz frequencies for radio waves\cite{ieee_wifi_standard}.


WiFi is ideal for indoor positioning because of its availability
\cite{indoor_maps_google_slides}\cite{improving_wifi_using_bluetooth}.
This is the primary reason for using WiFi as the other methods available, bluetooth and RFID, even though they are more accurate, are more expensive to set up \cite{improving_wifi_using_bluetooth}.

\paragraph{Limitations}
When measuring signal strength to WiFi receivers there is the following noise:
 \begin{itemize}
 	\item Interference by other devices using the 2,4 GHz frequency - cordless phones, bluetooth devices etc.
 	\item The 2,4 GHz frequency is a resonant frequency of water. This means that humans interferes with the signal as humans mostly consist of water.
\end{itemize}

Another challenge to take into account is the power usage.
A WiFi interface of a mobile phone consumes 5 times as much power of what bluetooth devices use\cite{bluetooth_basics}.
\subsubsection{Bluetooth}
The following is based on \citet{ieee_bluetooth_standard}.
Bluetooth is a low power consumption communications technology, intended for short-distance data exchange.
It is typically used to connect equipment to a device that is naturally close in proximity.
An example of this relation could be connecting a mouse or a keyboard to a computer or a mobile device.
Bluetooth is also used for handsfree headsets, for both mobile phones and landlines.
Such connectivity allows for both the transfer of audio as well as commands to the controlling unit.
This allows the headset to perform things such as accepting and ending calls.

\paragraph{Limitations}
The range of a Bluetooth device depends on its power consumption.
Because of this, devices are available with varying ranges.
The ranges vary from 1 meter to 100 meters \cite{bt}.
\subsubsection{RFID}
RFID (Radio Frequency Identification) provides a way to tag items by means of a cheap tag that can then later be read with a RFID reader.

A RFID tag is a small object often in the form of a sticker, that can hold data and sends it to a reader upon request.
A tag contains a microchip containing the data and an antenna for communication with a reader.
Tags come in different variants regarding power source.
There exists both passive and active RFID tags.\cite{rfidreview}

The passive tags have no power source, and therefore relies on electromagnetic energy from the reader.
This limits the abilities of the tag to only containing a small amount of information, but they are very cheap.
Also the range of these tags are limited. \cite{rfidreview}

Active tags one the other hand have an internal battery.
This enables longer read ranges, onboard memory and even processing capacities.
The battery shortens the lifespan of the tag which is typically between 5 and 10 years and they are more expensive than the passive counterpart. \cite{rfidreview}

\paragraph{RFID Readers}
RFID readers contains a transceiver (a device with both a receiver and a transmitter) and an antenna.
The task of the reader is to read and write data to RFID tags.
Collected data is transferred to a computer when the data is received from the tag.\cite{rfidreview}
\subsection{Methods}
This section explores three methods commonly used for radio-based technologies to provide localization: \textit{triangulation, proximity analysis, fingerprinting} and \textit{hybrid}.
\subsubsection{Triangulation}\label{triangulation}
Triangulation is the technique of using distance measurements to a number of reference points to calculate the location of an object.
Typically the distance is measured using the signal strength between object and the reference points or the time of arrival of the signal.
A model is then used to map the distance measurements to a 2D or 3D position of the object.\cite[Section 4.1]{rfidreview}

\paragraph{SpotON}
An implementation of triangulation is the SpotON system\cite{spoton}.
SpotON uses several base stations that are all connected to a central server.
The server aggregates the received positions, triangulates the data and sends the resulting position to clients. \cite{spoton}

The algorithm used for triangulating the positions is a hillclimbing algorithm as presented in \citet[Section 3.3.1]{spoton}.
The algorithm starts from a arbitrary position and searches for the real position of the object by calculating the theoretical signal strengths for adjacent positions in a coordinate system.
These theoretical signal strengths are calculated with a model for each base station based on empirical data.
The theoretical signal strengths are then compared to the observed signal strengths, and the point with the smallest error is used as the next point of origin.
This process is repeated until the point does not move \cite{spoton}.

\paragraph{Criticism}
This method of triangulation is dependent on the calibration of the model of each base station.
It is therefore prone to errors when the environment changes.
Moving the base stations or even the surrounding objects will influence the precision of the calculations.

Also the hillclimbing method can be a problem in big buildings or big rooms, if there is no way to provide a good guess for the initial position.
Here short ranged RFID tags used for room partitioning could be used in order to have a reasonable starting point.


\subsubsection{Proximity analysis}\label{rooms}
This section describes a simple method (proximity analysis) of performing localization using radio-based hardware.
The technique, as described in \citet[Section II.C Proximity]{wireless_survey}, works by determining if the item being located is within the range of one or more fixed antennas with known locations.
If the item is within the range of a single antenna, its location is said to be that of the antenna.
If the item is within the range of multiple antennas, its location is said to be that of the antenna with the greatest signal strength.

In \cref{rooms:fig:cover} are three rooms $R_1$, $R_2$ and $R_3$ that are fitted with low range radio-based antennas.
Rooms $R_1$ and $R_2$ have been fitted with antennas to allow for the localization whereas $R_3$ only has preexisting antennas.

\begin{figure}[h]
\centering
\tikzstyle{rfid}=[draw=none, fill=blue, fill opacity=0.7, text opacity=1, circle, minimum size=0.6cm]
\begin{tikzpicture}[scale=0.5]

\draw  (-6,0) rectangle (-12,4);
\node at (-11.5,3.5) {$R_1$};

\draw  (0,0) rectangle (-6,4);
\node at (-5.5,3.5) {$R_2$};

\draw  (0,0) rectangle (4,6);
\node at (0.5,5.5) {$R_3$};

\def \n {5}
\def \radius {3cm}
\def \margin {8} % margin in angles, depends on the radius

\foreach \x in {0,...,4}{\foreach \y in {0,...,2}{
  \node[rfid] at (-11 + \x,3 - \y) {};
}}
\foreach \x in {0,...,2}{
  \node[rfid,fill=green] at (-5 + \x * 2,3) {};
}
\foreach \x in {0,...,1}{
  \node[rfid,fill=green] at (-4 + \x * 2,2) {};
}
\foreach \x in {0,...,2}{
  \node[rfid,fill=green] at (-5 + \x * 2,1) {};
}

\node[rfid,fill=red] at (1.5,5.5) {};
\node[rfid,fill=red] at (3,3.5) {};
\node[rfid,fill=red] at (1.5,2) {};
\node[rfid,fill=red] at (3,1) {};

\end{tikzpicture}
\caption{Covering rooms with antennas}
\label{rooms:fig:cover}
\end{figure}

It is clear that the precision of this method is directly related to the density of antennas, as the number of antennas is equal to the number of possible positions returned by the system.
Thus the precision in rooms $R_1$ and $R_2$ is much greater than that in room $R_3$.
It might however be too expensive to provide all rooms with such a high density of antennas and having few antennas provides very rough location information.
Whenever the object being tracked is not in range of an antenna its location is unknown.
Because of this only the last known location can be provided when location is queried.

\paragraph{Splitting into rooms}
Instead of using proximity analysis to determine the \textit{location} of an item, it could instead be used to determine the room in which the item is located.
This method does not require a greater granularity than that which the method provides.
In \cref{rooms:fig:rooms} the same set of rooms as in \cref{rooms:fig:cover} is fitted with only four antennas.
The antennas are placed strategically at the doors between the rooms.
Because of this they can be used to determine if the item travels from one room to another.

\begin{figure}[h]
\centering
\tikzstyle{rfid}=[draw=none, fill=blue, fill opacity=0.7, text opacity=1, circle, minimum size=0.6cm]
\begin{tikzpicture}[scale=0.5]

\draw  (0,0) rectangle (4,6);
\node at (0.5,5.5) {$R_3$};

\draw  (0,0) rectangle (-6,4);
\node at (-5.5,3.5) {$R_2$};

\draw  (-6,0) rectangle (-12,4);
\node at (-11.5,3.5) {$R_1$};

\node[rfid] at (-6.5,2) {$i_0$};
\node[rfid,fill=green] at (-5.5,2) {$i_1$};

\node[rfid,fill=green] at (-0.5,2) {$i_2$};
\node[rfid,fill=red] at (0.5,2) {$i_3$};

\end{tikzpicture}
\caption{Covering doors between rooms with antennas}
\label{rooms:fig:rooms}
\end{figure}

If an item first passes by $i_0$ and then $i_1$ we know that it has moved from $R_1$ to $R_2$ and likewise for the opposite direction.
If the object only passes by a single antenna (e.g. $i_0$) and looses the connection again, we know that it remains in $R_1$.

This method requires only knowledge about which room the item starts in to work.
The accuracy within a room is very as it is reduced to \textit{''in the room or not in the room''}.
However the system will always be able to correctly identify which room the object is located in, given that the initial room is known.
\subsubsection{Fingerprinting}\label{fingerprinting}
This section is based on \citet{fingerprinting_slides} and \citet{fingerprinting}.
Location fingerprinting is a positioning technique that measures the strength of different signals at a given position.
This measurement is then the fingerprint for that location.

This section will explain the two phases of fingerprinting and how they are stored.
\paragraph{Offline phase}
In the offline phase a site-survey is performed, where a radio map is being build.
This means measuring the strength of the different signals at all points of interest and storing them in a radio map.
A radio map is a database that contains received signal strength indicator(RSSI) values for different positions throughout a building.
A position is represented by a triplet $(x,y,z)$ where $x$ and $y$ represent the 2-D coordinates and $y$ the height of the sensor that measures the signal.
The signals are then stored for instance as a vector for each position.

For instance if one is using WiFi for localization, when measuring a specific location the RSSI from each access point is saved and that is the fingerprint for that location.

\paragraph{Online phase}
The online phase consists of collecting the RSSI values from the client and look them up in the radio map and return the position.
But most of the time the values collected from the client do not match any values on the radio map.
For that one needs to use a position estimation method to get the best match.
The simplest is to use the Nearest Neighbour algorithm, that will return the position with the shortest vector distance.
Other algorithms useful are Weighed K Nearest Neighbour and K Nearest Neighbour.


\subsubsection{Hybrid}
With WiFi being so inaccurate and being so sensitive to noise, see section \ref{wifi}.
Some studies have suggested doing indoor mapping by merging technologies such as WiFi and bluetooth \cite{hybrid_wifi_bluetooth} \cite{fusion_wifi_bluetooth}.

The hybrid model suggested by \citet{hybrid_wifi_bluetooth} uses WiFi and bluetooth.
It argues that because of the availability of WiFi and the cost of deploying bluetooth devices in a large area are very high, a hybrid model is the best solution in most cases for improving the accuracy.

The model uses the fingerprinting technique mentioned in \cref{fingerprinting} on the WiFi and the room splitting technique mentioned in \cref{rooms} for bluetooth devices.
The room splitting technique can be used on both bluetooth and RFID, the only difference is different obtainable accuracies because of the hardware capabilities.
Using room splitting reduces the computations needed to estimate the position by the fingerprinting technique, by reducing the size of the radio map.
\section{Dead Reckoning}
Dead reckoning (DR) \cite{DR} is a navigation technique to estimate the positioning of a moving object for when signals of other technologies, like WiFi and bluetooth, are too weak to be read or blocked.
DR makes use of previously known positions to predict an estimated positioning of the object after a given time.
The known positions can be used to estimate the direction and velocity of the object over elapsed time if they are not already known.
Using the data it is possible to compute a displacement from the last known position after the time has passed.
The estimated displacement of the object is easily influenced by sources of error as the object might have changed direction or stopped moving.
This may result in a wrong estimate which may drift over time.

Data used for DR can be obtained from internal sensors that the object is carrying or external sensors which is observing the object.
An example which include both internal and external sensors could be smartphones.
Smartphones can be self-contained using built-in components such as gyroscope, accelerometer and compass.
They might also use the built-in GPS to communicate with satellites which provides the smartphones with external data.

\section{Geo-Magnetism}
Each building will have a unique indoor magnetic field, made unique by the construction (concrete, steel beams etc.) of each building.
Each individual location within a building will be affected by the aforementioned parameters and will therefore have a unique signature, depending on how the magnetic field is at that exact location.

There have been very few attempts at utilizing these properties of the unique magnetic fields for a building, most noteworthy is \cite{geomagnetism}.
In \citet{geomagnetism} experiments are carried out in two buildings with both a robot and a human-wearable compass-construction.
Simple electric compasses are used for collecting data in a full 360 degree radius, while the least root mean square is calculated and nearest neighbour search is performed to match obtained location data with obtained map data.
Initial results with the robot were $err_{mean} = 3.05 m$, $err_{sd} = 4.09 m$ and $err_{max} = 15 m$, where 70 \% of the predictions had errors less than 2 m.
The second experiment with a human-worn device (with 4 compasses) yielded an overall result of $<1m$ error for 75.7 \% of the results.
Two overall methods of improving the results are suggested, where one is adding an additional hardware component for assisting localization and the other being an added algorithmic.
By adding this filtering (by limiting search space based on previous known location) a $<1 m$ accuracy for 86.6 \% of predictions is obtained.
Finally experiments are carried out to show that the magnetic field changes little over time (very little noticeable difference with 6 months between) and that furniture and electrical equipment have little impact on measurements.




\section{Conclusion}
We have in this paper focused on indoor mapping, where we have studied different solutions.
There are multiple solutions using different technologies: WLAN, Bluetooth and RFID and methods: Triangulation, Proximity analysis, Fingerprinting and Hybrid.

The different radio-based technologies and localization methods are prone to errors due to changing environment and obstacles.
Technologies such as WLAN are widely common thus easily available to use for indoor mapping while Bluetooth and RFID are on the other hand not commonly available and are thus costly to implement.
The triangulation, proximity analysis and fingerprinting are different methods that uses mapping and reference points to estimate the position of the object.

The use of magnetic fields is, in theory, a good alternative to WiFi-based localization, as it is completely independent of fixed-location hardware, such as WiFi hotspots.
However, it does share the same limitation as the other fingerprinting methods, in the requirements of a pre-made offline map before online localization can be performed.
As can be seen, it is still relatively new and unexplored, and thus it requires further research before it can be completely deemed useful.



\bibliographystyle{unsrtnat}
\bibliography{bibliography}

\end{document}
